\documentclass{article}
\usepackage{geometry}[margin=5pt]
\usepackage{amsmath}
\usepackage{pifont}
\usepackage{array}
\newcommand{\xmark}{\ding{55}}
\usepackage[slovak]{babel}
\usepackage[T1]{fontenc}
\usepackage{graphicx}

\graphicspath{{images/}}
\setlength{\fboxrule}{0pt}
\def\sranda#1{
	\begin{minipage}{5em}
		\vspace{1em}
		\includegraphics[width=\textwidth]{#1}
		\vspace{0em}
	\end{minipage}
}

\pagestyle{empty}

\begin{document}
	\begin{table}
		\centering
		{\Huge Vzorce pre deriváty guľe}\\[4em]
		\Large
		\begin{tabular}{ | c | c | c | c | }
			\hline
			Teleso & Obrázok & V - Objem & S - Povrch \\
			\hline
			Guľa & \sranda{gula} & $\frac{4}{3}\pi r^3$ & $4\pi r^2$ \\
			\hline
			Guľový pás & \sranda{gulovy-pas} & \xmark & $2\pi rv$ \\
			\hline
			Guľový vrchlík & \sranda{gulovy-vrchlik} & \xmark & $2\pi rv$ \\
			\hline
			Guľový odsek & \sranda{gulovy-odsek} & $\frac{1}{3} \pi v^2(3r - v)$ & $2\pi rv + \pi r_1^2$  \\
			\hline
			Guľová vrstva & \sranda{gulova-vrstva} & $\frac{1}{6}\pi v(3r_1^2 + 3r_2^2 + v^2)$ & $2\pi rv + \pi r_1^2 + \pi r_2^2$ \\
			\hline
			Guľový výsek & \sranda{gulovy-vysek} & $\frac{2}{3}\pi vr^2$ & $\pi r(r_1+2v)$ \\
			\hline
		\end{tabular}
	\end{table}
\end{document}

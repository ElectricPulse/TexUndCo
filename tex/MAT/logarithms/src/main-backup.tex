\documentclass{article}
\usepackage{geometry}[margin=5pt]
\usepackage{amsmath}
\usepackage[slovak]{babel}
\usepackage[T1]{fontenc}

\title{Základné logaritmické pravidlá a ich dôkazy pre laikov}
\author{Adam Labuš}

\begin{document}
\begin{titlepage}
	\maketitle
\end{titlepage}

\large
\section{Úvod - Definícia logartimu}
Vytvorme hypotetickú funkciu log(), ktorá na vstup bere dve čísla ``a`` a ``x``. Hodnotu tejto funkcie nazveme ``y``. Súvis týchto troch premenných sa dá zapísať nasledovne: \\
\begin{align}
	a^y = x
\end{align}
Správanie našej funkcie vieme potom vyjadriť takto: \\
\begin{align}
	\log_a(x) = y
\end{align}
Dôvod prečo takúto funkciu potrebujeme je, lebo nam pomáha vyjadriť ``y`` z rovnice (1). Bohužiaľ inak ako pri umocňovaní, pravidlá logaritmov niesu na prvý pohľad také intuitívne, preto niektoré ``zjednodušenia`` alebo ``pravidlá`` treba podložiť dôkazom.


\section*{Pravidlo I.}
Z úvodu vyplýva: \\
\begin{align}
	\log_a(a^y) & = y \tag{dosadili sme si x}
\end{align}


\section*{Dôkaz pravidla II. - logaritmické pravidlo mocniny}
Toto pravidlo vzniklo asi tým, že niekto sa snažil inak vyjadriť hodnotu nasledovného výrazu: \\
\begin{align}
	\log_a((a^{y_1^y_2}) = ?
\end{align}
Tak poďme na to!: \\
\begin{align}
	\log_a((a^{y_1^y_2}) & = \log_a(a^{y_1 * y_2}) \\
	& = y_1 * y_2 \tag{použili sme I.}
\end{align}
V tomto pravidle ide o to, že pri jeho použití budeme poznať hodnotu ``$y_2$". Hľadať budeme "$y_1$". Preto si vyjadrime "$y_1$" a "$a^y_1$" niečim iným: \\
\begin{align}
	y_1 & = \log_a(a^{y_1}) \\
	a^{y_1} & = x \\
	\log_a(x^{y_2}) & = y_2 * \log_a(a^y_1) \\
	\log_a(x^{y_2}) & = y_2 * \log_a(x)
\end{align}
Aby sme dostali oficiálne znenie tohto pravidla, tak zmeníme názov písmenka ``$y_2$`` na ``n`` lebo stratilo svôj originálny význam: \\
\begin{align}
	\log_a(x^n) = n * \log_a(x^n)
\end{align}


\section*{Dôkaz pravidla III. - logaritmické pravidlo produktu}
Už budeme iba odvodzovať z predošlých pravidiel. V tomto dôkaze veľmi čudesne použijeme pravidlo 1. na to aby sme spojili dva logaritmy do jedného.\\
\begin{align}
\log_a(x_1) + \log_a(x_2) & = ? \\
	& = \log_a(a^{\log_a(x_1) + \log_a(x_2)})  \\
	& = \log_a(a^{\log_a(x_1) * a^(\log_a(x_2)}) \\
	& = \log_a(x_1 * x_2)
\end{align}

\section*{Dôkaz pravidla IV. - logaritmické pravidlo kvocientu}
To isté ako predošlí dôkaz lenže s odčítavaním.:
\begin{align} 
\log_a(x_1) - \log_a(x_2) & = ? \\
	& = \log_a(a^{\log_a(x_1) - \log_a(x_2)})  \\
	& = \log_a(a^{\log_a(x_1) - a^(\log_a(x_2)}) \\
	& = \log_a(x_1/x_2)
\end{align}

\section*{Dôkaz pravidla V. - logaritmické pravidlo zmeny základu logaritmu}
V tomto poslednom dôkaze použijeme. \\
\begin{align}
\log_a(x_1) & = ? \\
\log_a(x_1) & = log_a(x_2^{log_{x_2}(x_1)}) \\
\log_a(x_1) & = log_{x_2}(x_1) * log_a(x_2) \\
\frac{\log_a(x_1)}{\log_a(x_2)} & = log_{x_2}(x_1) \\
\log_{x_2}(x_1) & = \frac{\log_a(x_1)}{\log_a(x_2)}
\end{align}
Toto pravidlo je super, keď máme kalkulačku iba s jedným logaritmom ie. nedá sa meniť jeho základ.


\section{Záver}
Takže spolu v poradí sme potvrdili nasledovné:
\begin{align}
	& log_a(a^y) = y \\
	& log_a(x^n) = n * log_a(x) \\
	& log_a(x_1 * x_2) = log_a(x_1) + log_a(x_2) \\
	& log_a(x_1/x_2) = log_a(x_1) - log_a(x_2) \\
	& log_{x_2}(x_1) = \frac{\log_b(x_1)}{\log_b(x_2)}
\end{align}
\end{document}

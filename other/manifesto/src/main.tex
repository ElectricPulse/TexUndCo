\documentclass[11pt]{extarticle}
\usepackage[margin=50pt]{geometry}
\usepackage[slovak]{babel}
\usepackage[T1]{fontenc}

\title{Škriatok v školskom systéme}
\author{}
\date{}

\begin{document}
Pred začatím:
	- motivácia na písanie tohto textu: čas normálne venovaný učením sa na písomku som z frustrácie nad jeho bezvýznamovosťou radšej venoval písaním tohto textu.
	- pre férovosť: Gamča je super škola a problémy, ktoré opisujem sa snaží napraviť, aj keď vôbec nie cielene alebo kompletne.
	Nie cielene, lebo o tejto téme sa skoro vôbec nebaví.
	Nie kompletne, lebo veľa problémov v školách po slovensku sú na Gamči ešte horšie.
	- bias: naštevoval som ZŠ Ostredková, tri roky Gymnázium Metodova a sedem rokov Gymnázium Grösslingova (ďalej len Gamča).
	všetky tri su jedny z najlepších škol na slovensku
\maketitle

Skoro každý v nejakej forme neznáša školu, preto akákoľvek kritika na ňu je väčšinou odpísaná ako výhovorky. V tomto texte sa pokúsim prejsť cez baríeru týchto predsudkov a opísať stav aktuálneho školského systému.
Ďalší problém v takejto kritike, spočíva v tom, že sa človek rýchlo zasekne v nekonečnej diskusií o detailoch - anglický svet tento fenomén, sústredenia sa na nepodstatné detaily, pomenovali "bike-shedding".
Preto predtým ako začnem s vypisovaním kritiky na každý individuálny predmet začnem paradoxne s mojím záverom z týchto problémov.
Keď s mojím záverom nebudete súhlasiť tak si stačí prečítať zvyšok textu.

\section Záver:
Školský systém, tak ako iné inšitúcie, ako napríklad cirkev, spĺňajú dva rôzne účely: vzdelávanie populácie, dosadzovanie moci nad populáciou. Skoro každá "vláda", či už ľudový demokrat, komunisti, nacisti, kráľ využíval školstvo ako spôsob ovládania populácie. Táto moc vyplýva so samotného procesu vzdelávania - nakoľko človek má absolutnú kontrolu nad tým čo a ako bude niečo učiť. Moc spočíva v tom, že ten kto vzdeláva si môže vybrať akékoľvek metódy, fakty, pohľady na dané fakty.
Môže akokoľvek klamať, závadzať, prikrášlovať alebo dôležité veci vynechávať.

Bolo by super keby si naši vládcovia sa tejto moci čo najrýchlejšie zbavili a rozložili ju múdro ľudu, ale miesto toho si ju skoro celú históriu ponechali.
O školských osnovách a metódach sa nerobia referendá a nakoľko je to tak abstraktný koncept pre bežného rodiča, ktorý má len útlu predstavu o tom ako jeho dieťa v škole trávi čas, tak je to zriedka volebná téma.
Samotné žiak, ktorý tieto nedostatky vidí je ale nesvojprávny, nemôže voliť a školskému systému by mal byť poslušný.
Ďalšia centralizácia moci sa prejavuje tak, že školské vedenie a učitelia majú právne, finačne a časovo zaviazané ruky a nemôžu zmysluplne systematicky zmeniť výučbu a to ani na školskej úrovni, nie to krajskej alebo štátnej.

Tento záujem moci v školstve je ten škriatok, ktorý všetko ničí. Je to dôvod, prečo školstvo nevzdeláva deti efektívne skoro v každej zemy na svete.
O samotnej forme ako by vzdelávanie malo vyzerať sa môžeme baviť donekonečna, ale akonáhle sa nezbavíme tohto konfliktu záujmov tak tieto zmeny nikdy nepresadíme.
Problémy ktoré budem opisovať sú len symptómy tohto škriatka.
Tento škriatok je obzvlášť prehliadnuteľný a nezabiteľný nakoľko má legislatívny ako aj kultúrny a finančný aspekt:
	- legislatívny
	zlá legislatíva propagujúca zlé učebné metódy
	- kultúrny
	celá populácia dospelých ľudí, ktorích očakávania o tom čo je vlastne učenie je pokrivená výučbou a skúsenosťami s školským systémom za ich mladého veku
	- finančný
	nakoľko dopady vzdelávania sú natoľko abstraktné, tak je jeho dôležitosť podcenovaná

% TODO: Postupne dopisuj jeden predmet za druhym
\section Problémy
	V tejto sekcí sa budem snažiť rozpísať jednotlivé problémy v školstve, pričom každý z nich by sa dal rozpísať na osobitnú knihu, preto sa budem niekedy vyhýbať detailom.

	\subsection Predmet - Dejepis:
		Pri výučbe dejepisu ako každého iného predmetu, by sme sa najprv mali zamyslieť prečo ho treba vyučovať.
		% TODO: pridaj zdroje
		"Vedomie o histórií nám pomáha predpokladať budúcnosť, čo nám pomáha robiť lepšie rozhodnutia v prítomnosti"
		Takže cieľ je vyučovať takú časť minulosti, ktorá najviac vystihuje daný časový rozsah.
		Tento rozsah môže na začiatku byť tisícročie - čo zmamená, že človek vie opísať približne trajektóriu, myšlienky a spoločnosť z daného tisícročia.
		Postupne sa tento rozsah bude zmenšovať a tým lepšie bude vedieť daný žiak predpokladať budúcnosť.

		Najväčší dopad na efektivitu dosiahnutia tohto cieľu je teda výber časového rozsahu.
		Treba učiť od prítomnosti smerom do minulosti, alebo od dávnej minulosti do prítomnosti.
		Na slovensku sa vyučuje tým druhým spôsobom, čo má značnú nevýhodu toho, že antika je od nášho bežného života natoľko oddelená, že náš mozog ju pokladá za nedôležitú.
		Na to čo je nedôležité sa veľmi ľahko zabúda a preto sú spojitosti medzi jednotlivými obdobiami minimálne.
		Prečo to náš mozog tak robí? - no predsa nevidí súvis medzi tým čo sa učí a čo vidí teraz okolo seba.
		Súvis medzi obdobím druhej svetov vojny a terajškom je jednoznačný oproti súvisu pádu rímskej ríše ku terajšku.
		Jednoducho preskočením obdobia medzi tým čo učím a terajškom úplne prestrihne skutočný súvis udalostí a tým celé učenie stráca pointu.
		Ešte horšie kým sa žiak dostane ku prítomnosti (20 st.) je už tak znudený, že mozog znovu zahlási, že aj toto obdobie bude nedôležité a väčšinu výučby odignoruje.
		Toto nerobia len žiaci, ale aj učiteľia dejepisu (lebo aj tí boli raz žiakmi).
		Ako to viem? Veľmi jednoducho, za posledných 7 rokov čo sa učím dejepis vždy aspoň 45 minút týždenne sme ani jeden krát neprebrali koniec 20. storočia, ani nehovorím o začiatku 21. st.
		20. st. je evidentne skoro vždy prehliadnuté a odpísané oproti tým predošlím a to má fatálne následky.
		Aj keď je generácia žiakov, ktorá sa tích 8 (opravte ma koľko to reálne je) rokov minimálne jeden krát do týždňa 45 minút učila dejepis, tak väčšina z nich
		by znovu volila nacistov, alebo komunistov.
		Je to tá istá generácia, ktorá nevie o bezvýznamých vojnách, ktoré Amerika páchala všade po svete.
		Je to tá istá generácia, ktorá nerozozná v Putinovi len ďalšieho Ruského Tcára.
		je to generácia, ktorá nevidí, že dejepisno-teritoriálne založené vojny štýlu "toto územie patrí etnicky a historicky mne" sú bezvýznamné
		
		Za tích 7 rokov dejepisu som ani jeden krát nepočul rok 1969, alebo 1989.
		Len matne sa prešlo cez svetové vojny zatiaľ, čo antike sa venovalo času štedro.
		
		Náš škriatko znovu niečo pokazil, vytvoril 5 miliónov historických analfabetov, ktorý nevedia zaradiť prítomnosť a už vôbec nie predpokladať budúcnosť.
		Sú to ľudia, ktorí nevidia diktátorské črty v pravých diktátoroch, ktorý nevedia ako sa začínajú vojny a ktorí nevedia ako sa končia.
		Ako spoločnosť nás táto okolnosť robí slabšou a viac fragmentovanou.
\end{document}

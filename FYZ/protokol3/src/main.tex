\documentclass[11pt]{extarticle}
\usepackage[margin=50pt]{geometry}
\usepackage{amsmath}
\usepackage[slovak]{babel}
\usepackage[T1]{fontenc}
\usepackage{multicol}
\usepackage{graphicx}
\usepackage{caption}
\graphicspath{ {../assets/} }
\usepackage{tikz}
\usepackage{stackengine}
\usepackage{pgfplots}
\pgfplotsset{compat=newest}
\usepackage{csvsimple}
\usepackage{float}
\usepackage{multirow}
\usepackage{caption}
\pgfplotsset{
	table/search path={../assets/},
}

\newcommand{\Epsilon}{\mathcal{E}}

\title{%
Laboratórna práca č.1\\
Deformácia ťahom
}
\author{}
\date{}

\begin{document}
\maketitle
\large
\begin{tabular}{ll}
	Dátum merania: 3.10.2024 \hfill &Meno: Adam Labuš \\
	Spolupracovníci: Ivan Cabaj, Lukáš Emrich, Andrej Blažek &Trieda: Septima B \\
\end{tabular}
\vspace{1em}
\section{Namerané a vypočítané hodnoty}
\newcommand{\colname}[2]{{\LARGE $\frac{#1}{#2}$}}
\renewcommand{\arraystretch}{1.1}
\newcounter{csvrownum}

%Gumicka
\setcounter{csvrownum}{0}
\begin{minipage}{0.5\textwidth}
\begin{center}
	\captionof{table}{Pružný materiál - gumička}
	\begin{tabular}{|c|c|c|c|c|c|}
		\hline
		\addstackgap[1em]{Č.M.} & \colname{m}{kg} & \colname{F}{N} & \colname{l_o}{m} & \colname{l}{m} & \colname{\Epsilon}{\%}\\
		\hline
		\csvreader[
			late after line=\\
		]{../assets/meranie-gumicka.csv}{}
		{
			\stepcounter{csvrownum}
			\csvcoli & \csvcolii & \csvcoliii & 
			\ifnum\value{csvrownum}=1 {\multirow{21}{*}{\csvcoliv}} \fi &
			\csvcolv & \csvcolvi }
		\hline
	\end{tabular}
\end{center}
\end{minipage}
\hfill
%Nitka
\setcounter{csvrownum}{0}
\begin{minipage}{0.5\textwidth}
\begin{center}
	\captionof{table}{Krehký materiál - nitka}
	\begin{tabular}{|c|c|c|c|c|c|}
		\hline
		\addstackgap[1em]{Č.M.} & \colname{m}{kg} & \colname{F}{N} & \colname{l_o}{m} & \colname{l}{m} & \colname{\Epsilon}{\%}\\
		\hline
		\csvreader[
			late after line=\\
		]{../assets/meranie-nitka.csv}{}
		{
			\stepcounter{csvrownum}
			\csvcoli & \csvcolii & \csvcoliii & 
			\ifnum\value{csvrownum}=1 {\multirow{10}{*}{\csvcoliv}} \fi &
			\csvcolv & \csvcolvi }
		\hline
	\end{tabular}
\end{center}
\end{minipage}

\begin{figure}
\begin{minipage}{0.5\textwidth}
	\caption{Graf F od $\Epsilon$ - pružný materiál}
	\resizebox{\linewidth}{!}{
		\begin{tikzpicture}
			\begin{axis}[y label style={rotate=270}, xlabel={\Large $\frac{\Epsilon}{\%}$}, ylabel={\Large $\frac{F}{N}$}, ymin=0]
			\addplot table [x=col6, y=col3, col sep=comma] {meranie-gumicka-graf.csv};
		\end{axis}
		\end{tikzpicture}
	}
\end{minipage}
\hfill
\begin{minipage}{0.5\textwidth}
	\caption{F od $\Epsilon$ - krehký materiál}
	\resizebox{\linewidth}{!}{
		\begin{tikzpicture}
		\begin{axis}[y label style={rotate=270}, xlabel={\Large $\frac{\Epsilon}{\%}$}, ylabel={\Large $\frac{F}{N}$}, ymin=0]
			\addplot table [x=col6, y=col3, col sep=comma] {meranie-nitka-graf.csv};
		\end{axis}
		\end{tikzpicture}
	}
\end{minipage}
\end{figure}


\begin{figure}
\caption{F od $\Epsilon$ - pružný a krehký materiál}
\begin{center}
	\begin{tikzpicture}[scale=1.5]
		\begin{axis}[y label style={rotate=270}, xlabel={\Large $\frac{\Epsilon}{\%}$}, ylabel={\Large $\frac{F}{N}$}, ymin=0]
			\addplot table [x=col6, y=col3, col sep=comma] {meranie-nitka-graf.csv}
			node(plot1){};
			\node [right, color=blue] at (plot1) {\large Nitka};
			\addplot table [x=col6, y=col3, col sep=comma] {meranie-gumicka-graf.csv}
			node(plot2){};
			\node [left, color=red] at (plot2) {\large Gumička};
		\end{axis}
		\end{tikzpicture}
\end{center}
\end{figure}

\section{Záver}
Skúmali sme deformáciu ťahom pre dva rôzne materiály: gumičku a nitku. Gumička je zástupca pružných materiálov a nitka krehkých materiálov. Gumička sa zdeformovala pod maximalnou vonkajšou ťahovou silou $10.5N$. Nitka sa zdeformovala pod podobnou maximalnou vonkajšou ťahou silou $10N$. Rozdiel medzi v zdeformovaní je v hodnote maximálneho relatívneho predĺženia - pre gumičku sme namerali skoro $400\%$, zatiaľ čo pri nitke len $7\%$. Takže nitka sa skoro vôbec pri záťaži nezdeformovala, zatiaľ čo gumička skoro zpäťnásobila svoju dĺžku.

Na začiatku oboch grafov je lineárna krivka - tu platí priama úmernosť F ku $\Epsilon$ teda Hookov zákon. V predošlom grafe deformácie pružného materiálu (Obr. 1) vieme takisto vyznačiť charakteristické oblasti:\\

\newcommand{\vyznac}[2]{
\addplot[draw=none, fill=#2] table [
	col sep=comma,
	x expr=\thisrow{col6},
	y expr=\thisrow{col3},
	restrict expr to domain={\thisrow{col6}}{#1}
] {meranie-gumicka-graf.csv} \closedcycle;
}


\begin{figure}[H]
	\centering
	\caption{Graf F od $\Epsilon$ - pružný materiál - deformačné oblasti}
	\begin{tikzpicture}[scale=1.3]
		\begin{axis}[y label style={rotate=270}, xlabel={\Large $\frac{\Epsilon}{\%}$}, ylabel={\Large $\frac{F}{N}$}, ymin=0, ymax=12]
		\addplot table [x=col6, y=col3, col sep=comma] {meranie-gumicka-graf.csv};
		\vyznac{0:500}{red}
		\vyznac{0:320}{blue}
		\vyznac{0:180}{green}
			\node at (50, 5.5) [color=green] {\parbox{2cm}{\textbf{pružná\\deformácia}}};
			\node (1) at (180, 7) [color=blue] {tečenie};
			\node (2) at (280, 9) [color=red] {spevňovanie};
			\node (root) at (70, 10) [] {\parbox{2cm}{\textbf{plastická\\deformácia}}};
			\draw[->] (root.east) -- (1.north);
			\draw[->] (root.east) -- (2.west);
	\end{axis}
	\end{tikzpicture}
\end{figure}

\end{document}

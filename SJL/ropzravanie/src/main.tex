\documentclass[12pt, a4paper]{article}
\usepackage[margin=75pt]{geometry}
\usepackage[slovak]{babel}
\usepackage{csquotes}
\MakeOuterQuote{"}
\usepackage[T1]{fontenc}

\title{%
	Ten pohľad mi vyrazil dych - Rozprávanie
}
\author{Adam Labuš}

\begin{document}
	\maketitle
Na obzore bola vidieť pevnina, čo znamenalo, že na navigáciu stačilo už len sledovať viac a viac zaľudnené pobrežie.
Takto sme sa po hodine dostali ku zálivu, kde sa rieka Hudson vlievala do Atlantiku.
Ako prvú som uvidel Sochu slobody, hneď za tým sa otvoril pohľad na mesto, ktorý mi vyrazil dych:
Prístav posiaty skladmi a parníkmi - za tým výškové budovy a dlhé ulice, plné ľudí a krásnych áut.
My sme však smerovali na Ellis Island.
Tam prebehne hraničná kontrola, čo mi naháňalo husiu kožu.
Po vyčakaní rady sa ma dvojica hraničiarov vypytovala na formality: dôvod prisťahovania, pôvod a podobne.
Potom som podstúpil dôkladnej zdravotnej prehliadke a nakoniec mi úradníčka opečiatkovala pas, \enquote{Vitajte v Amerike.}

Potom nás prepravili kompou do prístavu, uľavilo sa mi, keď som po mesiaci konečne vkročil do ulíc New Yorku. So sebou som mal jeden kufor a po kúpe lodného lístka a zamenení korún, v peňaženke 25 dolárov. 
\enquote{To je slušná suma, mala by vystačiť aspoň na 3 týždne}, pomyslel som.

Po chvíľu blúdenia som sa objavil na prahu 3 poschodovej tehelnej budovy na kraji centra.
Podivný muž tam ponúkal jednu izbu v byte nad ním na prenájom za skvelú cenu.
Krátky rozhovor s prenajiteľom stačil na to, aby sme sa dohodli, \enquote{Ešte jedna vec, musíte zaplatiť polovicu nájomného ako zálohu.}
Toto mi prišlo trochu podozrivé, ale veľmi ma to netrápilo, nakoľko ako rodený obuvník sa aj tak isto do pár dní zamestnám.
\enquote{Dohodnuté}, oznámil som a podal mu 20 dolárovku.
\enquote{Idem pre Vaše klúče a výdavok}, odvetil a zabuchol dvere.
Pol hodina prešla a on sa stále nevrátil a ja som už tušil, že kľúče ako aj zálohu nikdy neuvidím. Ukrátim vás o detaily: Muž na danej adrese nebýval a žiadnu volnú izbu nevlastnil.
Hlúpy ja som si nevypýtal ani jeho meno alebo číslo, takže telefonát s políciou bol veľmi krátky. Pravdepodobne počuli môj cudzínecký prízvuk keď povedali, \enquote{Takéto veci sa dejú denne, treba si dávať pozor, lebo my s tým nič potom nenarobíme.}
Porazene som zavesil telefón, s vedomím akú Pandorinu skrinku problémov som práve otvoril.
Už sa blížil večer, s ním aj hlad, tak som si vytiahol zvyšok desiaty ktorý mi ostal v kabáte.
A tak s kufrom, chlebom v ruke a skoro prázdnou peňaženkou som sa s obdivom prechádzal cez neonovo rosvietenú ulicu Broadway.
Zastavil som sa, keď som uvidel rozložené stany v diere v zástavbe.
Boli tu počuť rôzne európske jazyky takisto ako v prístave.
V strede tábora bol zapálený oheň, pri ňom chlapík rozprával príbeh zopár ľuďom sediacemu ticho v kruhu okolo ohňa.
Popritom si popíjal z flaše whisky a raz za čas sa on aj obecenstvo spoločne zasmiali.
Bol neupravený a chudobne oblečený, čo mi dodávalo istotu, že nejako podobne sa živí ako ten podvodník na ktorého som dneska natrafil.
V tom svoje rozprávanie prerušil a prihovoril sa mi, \enquote{Hej ty zelenáč, neboj sa a poď bližšie.} So skúmavým pohľadom mi pri ohni uvolnili miesto a ja som si opatrne sadol medzi nich.
\\\enquote{Odkial si docestoval?}
\\\enquote{Z Uhorska.} odpovedal som s neistotou, či vôbec muž o ňom už niekedy počul.
\\\enquote{Fíha, takého tu ešte nemáme.
Nechaj ma hádať, rodina ostala v domovine a ty si došiel, lebo u vás hrozí ďalšia vojna a čul si o bohatsve Novej zemi, ale teraz nemáš ani kde spať.}
\\Zahanbene som pritakol.
Chytil ma za rameno. \enquote{Alee, hlavu vpred, všetci sme tu na tom rovnako. Môžem ti povedať to čo na imigračnom pohovore zamlčali? - Tráva tu nerastie o nič zelenšie. Vojnu tu síce nikdy nezažiješ, ale smrť budeš mať neustále na dotyk - tu v uliciach sa šíria choroby, chudoba a zlo. Nezabúdaj tu už človek nie je poddaný kráľovi, ale doláru a ten vie byť hrozne krutý...}

A takto začala naša debata, która pokračovala až dlho do noci.
Popritom bol počuť neprestávajucí šum mesta, ktorý prehlušil len občasný prechod električky, vnútri ktorej sedeli v tvári unavení ľudia vracajucí sa domov z práce. Niektorí špinaví od strojov, druhí zas oblečený v prepotených košeliach.
No každý z nich s tým istým snom konečného pokoja a teda bohatstva na mysli, tvoriaci si medzi sebou ukrutnú konkurenciu, lebo jedno, čo vám Socha slobody chce nanúkať, dosiahnúť slobodu môžu len najsilnejší z nich. Naraz mi pripadalo mesto ako mlynček na mäso.
Človek do neho vstúpi s morálkou a hodnotami z domoviny, no vystúpi zmenený, lebo jediné, čo má a môže obetovať pre vlastný úspech je samého seba.

Spomínal som na ten prvý úžas, keď som došiel, nad budúcnosťou, ktorá leží predo mňou, popri tom ako som s hlavou podopretou kufrom a prikritý kabátom, zaspával na chladnej zemy chodníka.
\end{document}
